\documentclass[journal,12pt,twocolumn]{IEEEtran}
\usepackage[utf8x]{inputenc}
\usepackage{listings}
\usepackage[none]{hyphenat}
\usepackage{enumitem}
\usepackage{graphicx}
\usepackage{kvmap}


\title{Assignment \textrm{I} \textbf{\\Simplifying Boolean expression using Kmap}}
\author{Manideep Parusha - FWC22004}
\date{\today}
\begin{document}
\maketitle

\tableofcontents

\section{Problem}
Reduce the following Boolean expression in the simplest form using Kmap. The Expression with Sum of Products (SOP) is as follows:
$$ {F(P,Q,R,S) = \sum (0,1,2,3,5,6,7,10,14,15)}$$

\section{Solution}

\subsection{Truth Table}
Truth table for the SOP given:
\begin{table}[h!]
    \centering
    
    \begin{tabular}[20pt]{|c|c|c|c|c|c|c|}
          \hline
          &P&Q&R&S&F(P,Q,R,S) 
          \\ \hline & & & & & 
          \\ 0&0&0&0&0&1
          \\ 1&0&0&0&1&1
          \\ 2&0&0&1&0&1
          \\ 3&0&0&1&1&1
          \\ 4&0&1&0&0&0
          \\ 5&0&1&0&1&1
          \\ 6&0&1&1&0&1
          \\ 7&0&1&1&1&1
          \\ 8&1&0&0&0&0
          \\ 9&1&0&0&1&0
          \\ 10&1&0&1&0&1
          \\ 11&1&0&1&1&0
          \\ 12&1&1&0&0&0
          \\ 13&1&1&0&1&0
          \\ 14&1&1&1&0&1
          \\ 15&1&1&1&1&1     
          \\ \hline
              \end{tabular}
            \vspace{9pt}
    \caption{Truth Table for given Boolean expression}
    \label{Truthtable}
\end{table}

\subsection{K-map}
K-map for the above truth table:\\
\begin{table}[h]

\centering
\begin{kvmap}
\kvlist{4}{4}{1,1,1,1,0,1,1,1,0,0,1,1,0,0,0,1}{P, Q, R, S}
%\bundle {3}{2}{2}{3}
\end{kvmap}
\vspace{4pt}
\caption{K-Map from the Truth Table}
    \label{kmap}
\end{table}

\subsection{Rules to simplify K-maps}
The Karnaugh map uses the following rules for the simplification of expressions by grouping together adjacent cells containing ones
\vspace{4pt}
\begin{enumerate}[itemsep=5pt]
    \item Groups may not include any cell containing a zero
    \item Groups may be horizontal or vertical, but not diagonal
    \item Groups must contain 1, 2, 4, 8, or in general $2^n$ cells
    \item Each group should be as large as possible.
    \item Each cell containing a one must be in at least one group
    \item Groups may overlap
    \item Groups may wrap around the table. The leftmost cell in a row may be grouped with the rightmost cell and the top cell in a column may be grouped with the bottom cell 
    \item There should be as few groups as possible, as long as this does not contradict any of the previous rules
\end{enumerate}


\subsection{Simplification}
\begin{table}[h]

\centering
\begin{kvmap}
\kvlist{4}{4}{1,1,1,1,0,1,1,1,0,0,1,1,0,0,0,1}{P, Q, R, S}
\bundle[color = red, reducespace = 1pt] {0}{0}{3}{0}
\bundle[color = black,reducespace = 3pt] {1}{0}{2}{1}
\bundle[color = green,reducespace = 2pt] {2}{1}{3}{2}
\bundle[color = blue] {3}{0}{3}{3}
\end{kvmap}
\vspace{4pt}
\caption{Grouped K-map}
    \label{kmap_grouped}
\end{table}

Simplified Boolean expression using Kmap without don't care will be:
$$F(P,Q,R,S) = P'Q' + P'S + RS' + QR$$

\section{Verification \& Conclusion}
The simplified Boolean expression can be implemented using 

\vspace{10pt}
\begin{tabular}{|c|}
    \hline
wget https://raw.githubusercontent.com/parusamanideep
\\/FWC/main/assignment1/src/main.cpp
     \\ \hline
\end{tabular}

\vspace{4pt}
The file in the above link is a C program to implement the logic thet is simplified using the K-map.
\\ The below link has the assembly program to implement the simplified logic using K-map.

\vspace{10pt}
\begin{tabular}{|c|}
    \hline
wget https://raw.githubusercontent.com/parusamanideep
\\/FWC/main/assignment1/asm/Assignment1.asm
     \\ \hline
\end{tabular}
\bigskip
\\ P, Q, R, S are given as inputs to 2, 3, 4, 5 pins respectively from the 5V and GND lines.
\\The given Boolean expression is simplified and verified for functionality.


\end{document}
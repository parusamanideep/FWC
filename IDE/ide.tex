\documentclass[journal,12pt,twocolumn]{IEEEtran}
\usepackage[utf8x]{inputenc}
\usepackage{listings}
\usepackage[none]{hyphenat}
\usepackage{enumitem}
\usepackage{graphicx}
\usepackage{kvmap}


\title{IDE - Assignment \textrm{I} \textbf{\\Simplifying Boolean expression using Kmap}}
\author{Manideep Parusha - FWC22004}
\date{\today}
\begin{document}
\maketitle

\tableofcontents 
\bigskip
\begin{abstract}
This manual describes the rules for the simplification of K-maps and steps to develop the simplified expression which is furthuer implemented on hardware for functionality for a given Boolean Function. 
\end{abstract}
\section{Problem}
Reduce the following Boolean expression in the simplest form using Kmap. The Expression with Sum of Products (SoP) is as follows:
$$ {F(P,Q,R,S) = \sum (0,1,2,3,5,6,7,10,14,15)}$$

\newpage
\section{Solution}

\subsection{Truth Table}
Truth table for the given Boolean function:
\begin{table}[h]
    \centering
    \begin{tabular}[20pt]{|c|c|c|c|c||c|c|}
          \hline
          &P&Q&R&S&F(P,Q,R,S) \\
          \hline
          0&0&0&0&0&1\\
          1&0&0&0&1&1\\
          2&0&0&1&0&1\\
          3&0&0&1&1&1\\
          4&0&1&0&0&0\\
          5&0&1&0&1&1\\
          6&0&1&1&0&1\\
          7&0&1&1&1&1\\
          8&1&0&0&0&0\\
          9&1&0&0&1&0\\
          10&1&0&1&0&1\\
          11&1&0&1&1&0\\
          12&1&1&0&0&0\\
          13&1&1&0&1&0\\
          14&1&1&1&0&1\\
          15&1&1&1&1&1\\   
          \hline 
              \end{tabular}
              \bigskip
    \caption{Truth Table}
    \label{Truthtable}
\end{table}
\subsection{K-map}
A Karnaugh map (K-map) is a visual method used to simplify the algebraic expressions in Boolean functions without having to resort to complex theorems or equation manipulations.  It involves fewer steps than the algebraic minimization technique to arrive at a simplified expression. K-map simplification technique always results in minimum expression if carried out properly. Rules to follow while simplifying a K-map mention in further sections. \\ 
\newpage
K-map for the above mentioned truth table:
\begin{table}[h]
\centering
\begin{kvmap}
\kvlist{4}{4}{1,1,1,1,0,1,1,1,0,0,1,1,0,0,0,1}{P, Q, R, S}
\end{kvmap}
\vspace{4pt}
\caption{K-Map from the Truth Table}
    \label{kmap}
\end{table}
\subsection{Rules to simplify K-maps}
The Karnaugh map uses the following rules for the simplification of expressions by grouping together adjacent cells containing ones
\vspace{4pt}
\begin{enumerate}[itemsep=5pt]
    \item Groups may not include any cell containing a zero
    \item Groups may be horizontal or vertical, but not diagonal
    \item Groups must contain 1, 2, 4, 8, or in general $2^n$ cells
    \item Each group should be as large as possible.
    \item Each cell containing a one must be in at least one group
    \item Groups may overlap
    \item Groups may wrap around the table. The leftmost cell in a row may be grouped with the rightmost cell and the top cell in a column may be grouped with the bottom cell 
    \item There should be as few groups as possible, as long as this does not contradict any of the previous rules
\end{enumerate}
\newpage
\subsection{Simplification}
\begin{table}[h]
\centering
\begin{kvmap}
\kvlist{4}{4}{1,1,1,1,0,1,1,1,0,0,1,1,0,0,0,1}{R,S,P,Q}
\bundle[color = red, reducespace = 1pt] {0}{0}{3}{0}
%\bundle[color = black,reducespace = 3pt] {1}{0}{2}{1}
%\bundle[color = green,reducespace = 2pt] {2}{1}{3}{2}
%\bundle[color = blue] {3}{0}{3}{3}
\end{kvmap}
\vspace{4pt}
\caption{K-map 1}
    \label{kmap_grouped1}
\end{table}
Resulting expression from the groups in the above K-map: P'Q'\\

\bigskip
\begin{table}[h]
\centering
\begin{kvmap}
\kvlist{4}{4}{1,1,1,1,0,1,1,1,0,0,1,1,0,0,0,1}{R,S,P,Q}
\bundle[color = red, reducespace = 1pt] {0}{0}{3}{0}
\bundle[color = black,reducespace = 3pt] {1}{0}{2}{1}
%\bundle[color = green,reducespace = 2pt] {2}{1}{3}{2}
%\bundle[color = blue] {3}{0}{3}{3}
\end{kvmap}
%\vspace{4pt}
\caption{K-map 2}
    \label{kmap_grouped2}
\end{table}
Resulting expression from the groups in the above K-map:  P'Q' + P'S\\
\newpage
\begin{table}[h!]
\centering
\begin{kvmap}
\kvlist{4}{4}{1,1,1,1,0,1,1,1,0,0,1,1,0,0,0,1}{R,S,P,Q}
\bundle[color = red, reducespace = 1pt] {0}{0}{3}{0}
\bundle[color = black,reducespace = 3pt] {1}{0}{2}{1}
\bundle[color = green,reducespace = 2pt] {2}{1}{3}{2}
%\bundle[color = blue] {3}{0}{3}{3}
\end{kvmap}
\vspace{4pt}
\caption{K-map 3}
    \label{kmap_grouped3}
\end{table}
Resulting expression from the groups in the above K-map: P'Q' + P'S + RS'\\

\begin{table}[h]
\centering
\begin{kvmap}
\kvlist{4}{4}{1,1,1,1,0,1,1,1,0,0,1,1,0,0,0,1}{R,S,P,Q}
\bundle[color = red, reducespace = 1pt] {0}{0}{3}{0}
\bundle[color = black,reducespace = 3pt] {1}{0}{2}{1}
\bundle[color = green,reducespace = 2pt] {2}{1}{3}{2}
\bundle[color = blue] {3}{0}{3}{3}
\end{kvmap}
%\vspace{4pt}
\caption{K-map 4}
    \label{kmap_grouped4}
\end{table}

Resulting expression from the groups in the above K-map: P'Q' + P'S + RS' + QR\\

As we have grouped all the 1's in the K-map, we can consider the obtained expression to be final. In the K-map above, we can group atmost four 1's only. So we cannot further simplify the obtained expression from the K-map.\\

\bigskip Hence, the simplified Boolean expression using Kmap will be: 
$$\textbf{F(P,Q,R,S) = P'Q' + P'S + RS' + QR}$$
\newpage
\section{Hardware Implementation} 

\subsection{Components required}
\begin{table}[h]
\centering
\begin{tabular}{|c|c|}
\hline
\textbf{Component } & \textbf{Qty.}\\
\hline
Arduino UNO & 1\\
\hline
Jumper wires & 4 \\
\hline
\end{tabular}
\bigskip
\caption{components list}
\end{table}

The hardware we are using for implementation is the Arduino UNO board, which has AtMega328p micro-controller on-board. It is an AVR micro-controller with 8-bit ALU, works at 16MHz clock frequency, 3 ports including 1 analogue port. \\
We are using 4 pins(2,3,4,5) of PORTD for input and 1 pin(13) of PORTB for output.

\subsection{Connections}
The following table shows the inputs to be given to the microcontroller for implementing the above simplified Boolean expression.\\

\bigskip
\begin{table}[h]
\centering
\begin{tabular}{|c|c|c|c|c|}
\hline
Arduino & 2 & 3 & 4 & 5 \\
\hline
Inputs & P & Q & R & S \\
\hline
\end{tabular}
\bigskip
\caption{Input connections}
\end{table}

P, Q, R, S are the inputs which can be set or reset by manually connecting them to 5V or GND respectively. In this way we can make 16 combinations of the input sequence and they can be tested for functionality.
The output is from from Pin 13, which drives the on-board LED of Arduino UNO. When the output is '1', the LED is 'ON' and when the output is '0', the LED will be 'OFF'.\\
 
\subsection{Code}
The code for implementing the logic of the Boolean expression in 'C++' using ide (platformIO) can be downloaded by the command mentioned below \\ \\
\begin{tabular}{|c|}
    \hline
wget https://raw.githubusercontent.com/parusamanideep
\\FWC/main/IDE/codes/src/main.cpp\\
     \hline
\end{tabular}\\

\bigskip
The logic in the code uses 4 AND operations and 3 OR operations for implementation. line of the code is shown below which computes the Boolean Expression simplified above.
$$ F = (!P\&\&!Q) || (!P\&\&S) || (R\&\&!S) || (Q\&\&R); $$
Few Arduino functions used in the code are listed below with there usage explained.\\

\begin{table}[h]
\centering
\begin{tabular}{|c|c|}
\hline
Syntax & use\\
\hline
pinMode(*,*)& set the pin as INPUT / OUTPUT\\
\hline
digitalWrite(*,*)& Output the given value to Pin\\
\hline
digitalRead(*)& Read the input from given Pin\\
\hline 
\end{tabular}
\bigskip
\caption{Arduino functions used in code} 
\end{table}  

\section{Conclusion}
The Simplified Boolen Expression is implemented on Arduino and it's functionality can be tested by connecting P, Q, R, S inputs to 5V and GND to give the Pin '1' or '0' value.\\
Total number of combination of inputs that can be formed are 16 and the output can be verified by comparing the truthtable and the output obtained for the given inputs.

\end{document}